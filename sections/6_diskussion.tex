\section{Diskussion}

\textit{[I diskussionen ska ni tolka och reflektera över era resultat, metoder och design. Detta är ett viktigt kapitel som visar er förmåga att kritiskt analysera ert eget arbete.]}


\subsection{Resultatdiskussion}

\textit{[Diskutera era resultat i relation till teori och era mål]}

\subsubsection{Uppfyllelse av mål}

Projektets huvudmål var att [upprepa mål från inledningen]. Resultaten visar att [diskutera hur väl målen uppfyllts].

\textbf{Mål 1:} [Mål] har [uppfyllts/delvis uppfyllts/ej uppfyllts] eftersom... Testresultaten från iteration 3 visar att [diskutera resultat i relation till målet].

\textbf{Mål 2:} ...


\subsubsection{Användbarhet}

Sett till Nielsens fem användbarhetskriterier \cite{nielsen2012} kan följande konstateras:

\textbf{Lärbarhet:} Resultaten visar att [diskutera]. I iteration 3 kunde [X]\% av användarna genomföra uppgift 1 utan hjälp, vilket tyder på...

\textbf{Effektivitet:} Den genomsnittliga tiden för... Detta kan jämföras med [referenspunkt om sådan finns]. Tiden är [acceptabel/lång] eftersom...

\textbf{Fel:} De mest förekommande felen var... Detta indikerar att designen [diskutera]. Problemet skulle potentiellt kunna lösas genom att...

\textbf{Tillfredsställelse:} SUS-poängen på [XX] är [över/under] genomsnittet för denna typ av system. Detta tyder på att användarna...


\subsubsection{Designbeslut i relation till teori}

\textit{[Diskutera hur era designbeslut relaterar till teori]}

Beslutet att placera [element] i [position] visade sig vara framgångsrikt/problematiskt. Detta kan förklaras med hjälp av [teori] som säger att... Våra resultat stödjer/motsäger detta genom att...

Tillämpningen av [designprincip] i [kontext] resulterade i... Detta överensstämmer med [författares] rekommendation om att...


\subsection{Metoddiskussion}

\textit{[Reflektera kritiskt över era metodval]}

\subsubsection{Styrkor och svagheter}

\textbf{Intervjumetodik:}

Val av semistrukturerade intervjuer var lämpligt eftersom det gav... En styrka med metoden var... En svaghet var att... I efterhand skulle [alternativ metod] potentiellt kunnat ge...

\textbf{Användbarhetstester:}

Antalet testdeltagare ([antal]) är [tillräckligt/begränsat] för att... Nielsen menar att 5 användare hittar cirka 85\% av användbarhetsproblemen \cite{nielsen2012}, vilket antyder att... Dock kan det argumenteras att...


\subsubsection{Alternativa tillvägagångssätt}

Ett alternativt tillvägagångssätt hade varit att... Detta hade kunnat ge... men valdes bort eftersom... I efterhand kan det konstateras att...


\subsubsection{Reliabilitet och validitet}

\textit{[Diskutera trovärdighet och tillförlitlighet]}

Studiens resultat kan betraktas som [reliabla/delvis reliabla] eftersom... För att öka reliabiliteten kunde...

Validiteten i studien stärks av att [beskriv styrkor]. Dock finns begränsningar såsom [beskriv svagheter], vilket innebär att...


\subsection{Etisk reflektion}

\textit{[Diskutera hur ni förhållit er till forskningsetiska principer]}

Projektet har genomförts i enlighet med Vetenskapsrådets forskningsetiska principer \cite{vetenskapsradet2002}.

\textbf{Informationskravet:} Alla deltagare informerades om [vad de informerades om]. Detta säkerställdes genom...

\textbf{Samtyckeskravet:} Samtycke inhämtades [skriftligt/muntligt] innan... Deltagarna informerades om att deltagandet var frivilligt och att de kunde avbryta när som helst.

\textbf{Konfidentialitetskravet:} All insamlad data har behandlats konfidentiellt genom att... Personuppgifter har [hur de hanterats].

\textbf{Nyttjandekravet:} Data har endast använts för detta projekt.

En etisk utmaning som uppstod var [beskriv om tillämpligt]. Detta hanterades genom att...


\subsection{Kritisk reflektion}

\textit{[Var kritisk mot ert eget arbete - men balanserat]}

\subsubsection{Begränsningar}

Projektet har flera begränsningar som påverkar generaliserbarheten av resultaten:

\begin{itemize}
    \item Stickprovsstorlek: [Antal] deltagare är en begränsning eftersom...
    \item Urval: Deltagarna rekryterades genom [metod], vilket kan ha lett till...
    \item Tidsram: Den begränsade tidsramen innebar att... Med mer tid hade...
    \item Prototypnivå: Eftersom endast en [lo-fi/hi-fi]-prototyp utvecklades...
\end{itemize}


\subsubsection{Vad gjorde vi bra?}

Trots begränsningarna finns flera styrkor i projektet:

\begin{itemize}
    \item Den iterativa processen med [antal] iterationer möjliggjorde...
    \item Användningen av flera metoder (triangulering) stärker...
    \item Nära involvering av målgruppen genom hela processen...
\end{itemize}


\subsubsection{Vad kunde förbättrats?}

I efterhand finns flera områden som kunde förbättrats:

\begin{itemize}
    \item Mer strukturerad analys av [data] genom att...
    \item Fler deltagare i [fas] hade kunnat ge...
    \item Bättre dokumentation av [aspekt]...
\end{itemize}


\subsection{Slutsats}

\textit{[Sammanfatta projektets huvudsakliga bidrag och slutsatser]}

Detta projekt har visat att [huvudsaklig slutsats]. Genom en användarcentrerad designprocess har ett gränssnitt för [målgrupp] utvecklats som [beskriv resultat].

De huvudsakliga bidragen är:
\begin{itemize}
    \item Ett användargränssnitt som [beskriv]
    \item Insikter om [målgruppens] behov vad gäller [område]
    \item Empiriskt stöd för att [designprincip/teori] är applicerbar på...
\end{itemize}

Resultaten visar att [sammanfatta viktiga fynd]. Detta indikerar att [tolkning].


\subsection{Framtida arbete}

\textit{[Vad skulle kunna göras om projektet fortsatte?]}

Med mer tid och resurser skulle följande kunna genomföras:

\textbf{Kortfristigt (1 månad):}
\begin{itemize}
    \item Implementera [funktion] som identifierades som önskvärd men prioriterades bort
    \item Utöka användbarhetstester till att inkludera [fler deltagare/andra målgrupper]
    \item Förbättra [specifik vy/funktion] baserat på feedback
\end{itemize}

\textbf{Långfristigt (6 månader):}
\begin{itemize}
    \item Implementera en funktionell prototyp i [teknologi]
    \item Genomföra longitudinella studier för att utvärdera [aspekt] över tid
    \item Expandera till andra plattformar [mobil/webb/...]
    \item Integrera med [existerande system]
\end{itemize}

Det skulle också vara intressant att undersöka [forskningsfråga] eftersom...
