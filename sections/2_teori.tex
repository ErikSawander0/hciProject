\section{Teori}

\textit{[Teorikapitlet ska lyfta de teoretiska delar som är relevanta för ert arbete. Fokusera på koncept från kurslitteraturen och forskningsartiklar som är viktiga för just ert projekt. Teorikapitlet ska vara tungt refererat till tidigare arbeten.]}

\textit{[Kom ihåg: Behandla endast teori som är relevant för ert arbete. Ni skriver inte en lärobok, utan lyfter bara de delar som ni senare kommer att referera till i design, metod eller diskussion.]}


\subsection{Användarcentrerad design}

\textit{[Exempel på teorisektion]}

Användarcentrerad design (User-Centered Design, UCD) är en designfilosofi där användaren står i centrum genom hela utvecklingsprocessen \cite{sharp2019}. Metoden innebär att användarna aktivt involveras i designprocessen genom iterationer av design, test och revidering.

De fyra huvudprinciperna för användarcentrerad design är:
\begin{enumerate}
    \item Tidigt fokus på användare och uppgifter
    \item Empirisk mätning av produktanvändning
    \item Iterativ design
    \item Integrerad design
\end{enumerate}


\subsection{Användbarhet}

\textit{[Beskriv relevanta användbarhetsbegrepp]}

Nielsen definierar användbarhet genom fem komponenter \cite{nielsen2012}:
\begin{itemize}
    \item \textbf{Lärbarhet}: Hur lätt är det för användare att utföra grundläggande uppgifter första gången?
    \item \textbf{Effektivitet}: När användarna lärt sig designen, hur snabbt kan de utföra uppgifter?
    \item \textbf{Minnesvärdhet}: När användare återkommer efter en period, hur lätt återupprättar de kompetensen?
    \item \textbf{Fel}: Hur många fel gör användare, hur allvarliga är de, och hur lätt kan de återhämta sig?
    \item \textbf{Tillfredsställelse}: Hur trevlig är designen att använda?
\end{itemize}


\subsection{Forskningsetiska principer}

\textit{[Detta är obligatoriskt att inkludera]}

I projekt som involverar människor är det viktigt att följa Vetenskapsrådets forskningsetiska principer \cite{vetenskapsradet2002}. De fyra huvudkraven är:

\begin{itemize}
    \item \textbf{Informationskravet}: Forskaren ska informera deltagare om forskningens syfte
    \item \textbf{Samtyckeskravet}: Deltagare har rätt att själva bestämma över sin medverkan
    \item \textbf{Konfidentialitetskravet}: Uppgifter om deltagare ska förvaras på ett sätt så obehöriga inte kan ta del av dem
    \item \textbf{Nyttjandekravet}: Insamlade uppgifter får endast användas för forskningsändamål
\end{itemize}


\subsection{[Lägg till fler teorisektioner efter behov]}

\textit{[Till exempel: Gestaltlagar, designprinciper, prototypmetoder, evalueringsmetoder, etc.]}
