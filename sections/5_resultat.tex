\section{Resultat}
\label{sec:resultat_test}

\textit{[Resultatkapitlet presenterar det konkreta resultatet från era intervjuer, användbarhetstester och andra utvärderingar. Presentera data OBJEKTIVT - ta ingen ställning eller tolka här. Det gör ni i diskussionen.]}

\textit{[Använd figurer, diagram och tabeller för att presentera data tydligt. Förklara diagram så läsaren förstår, men tolka inte ännu.]}


\subsection{Resultat från intervjuer}

\textit{[Presentera resultat från första intervjuomgången]}

Totalt intervjuades [antal] personer från målgruppen. Demografisk data presenteras i tabell \ref{tab:demografi}.

\begin{table}[h]
\centering
\begin{tabular}{|l|c|}
\hline
\textbf{Demografisk variabel} & \textbf{Antal/Medel} \\
\hline
Totalt antal deltagare & X \\
Ålder (medel) & XX år \\
Kön (M/K/Annan) & X/X/X \\
Erfarenhet av [relevant område] & ... \\
\hline
\end{tabular}
\caption{Demografisk data för intervjudeltagare}
\label{tab:demografi}
\end{table}


\subsubsection{Identifierade teman}

Genom tematisk analys identifierades följande huvudteman:

\textbf{Tema 1: [Temats namn]}

[Antal] av deltagarna nämnde problem relaterade till [beskriv]. En deltagare uttryckte: \textit{"[citat]"}. Liknande upplevelser rapporterades av...

\textbf{Tema 2: [Temats namn]}

[Fortsätt beskriva teman objektivt]


\subsubsection{Identifierade behov}

Baserat på intervjudata identifierades följande användarbehov:
\begin{itemize}
    \item Behov 1: [Beskriv]
    \item Behov 2: [Beskriv]
    \item ...
\end{itemize}


\subsection{Resultat från användbarhetstester - Iteration 2}

Femton användare deltog i användbarhetstester av prototyp version 2. Varje deltagare fick genomföra 4 uppgifter, men enkäten hade inte frågar om fjärde uppgiften. 


\subsubsection{Kvantitativa resultat}

\textbf{Genomförandetid}

Tabell \ref{tab:tid} visar tid för att genomföra varje uppgift.

\begin{table}[h]
\centering
\begin{tabular}{|l|c|c|c|}
\hline
\textbf{Uppgift} & \textbf{Medeltid (s)} & \textbf{Min (s)} & \textbf{Max (s)} \\
\hline
Uppgift 1:  & 4 & 1 & 12 \\
Uppgift 2:  & 24 & 4 & 60\\
Uppgift 3:  & 23 & 1 & 60\\
\hline
\end{tabular}
\caption{Tid för genomförande av uppgifter}
\label{tab:tid}
\end{table}

\textbf{Framgångsgrad}

Figur \ref{fig:success_rate} visar hur många deltagare som lyckades genomföra varje uppgift utan hjälp.

\begin{figure}[h]
    \centering
    % \includegraphics[width=0.7\textwidth]{bilder/success_rate.png}
    \caption{Framgångsgrad för varje uppgift (antal användare som klarade uppgiften utan hjälp)}
    \label{fig:success_rate}
\end{figure}

\textbf{Identifierade problem}

Totalt identifierades 20 unika användbarhetsproblem. Tabell \ref{tab:problem} sammanfattar de mest förekommande problemen.
%MARK make sure task descriptions mentioned, how we didnt include four...
\begin{table}[h]
\centering
\begin{tabular}{|p{6cm}|c|c|}
\hline
\textbf{Problem} & \textbf{Antal som drabbades} & \textbf{Allvarlighetsgrad} \\
\hline
Problem 1: Colors misleading or hard to look at & 6/15 & Medel\\
Problem 2: Time scheduling difficult & 2/15 &  Låg \\
Problem 3: Getting to edit profile unintuitive & 6/12 & Hög\\
\hline
\end{tabular}
\caption{Identifierade användbarhetsproblem och deras allvarlighetsgrad}
\label{tab:problem}
\end{table}


\subsubsection{Kvalitativa resultat}

Under testerna observerades följande beteenden:

\textbf{Navigation:} Flera användare (6/15) uttryckte förvirring när de skulle redigera en profil, det var många som råkade börja brygga kaffe när de försökte redigera.

\textbf{Navigation:} 3 användare noterade att knapparna var förivrande, och att de skulle uppskattat mer symboler. 

\textbf{Färger} 6 användare klagade att färgerna var inte konsekventa eller hade problem med kontrasten. 


\subsubsection{Subjektiv tillfredsställelse}

Efter testerna fick deltagarna svara på hur svåra de tyckte var uppgift var att genomföra. Medelvärdet var 68 av 70.

% we didnt do this sus point thing... update method erik
\begin{figure}[h]
    \centering
    % \includegraphics[width=0.6\textwidth]{bilder/sus_score.png}
    \caption{SUS-poäng från deltagare}
    \label{fig:sus}
\end{figure}


\subsection{Resultat från användbarhetstester - Iteration 3}

\textit{[Om ni gjorde fler iterationer, presentera även dessa resultat. Visa förbättringar jämfört med tidigare iteration.]}

Efter designändringar (beskrivna i avsnitt 4.3) genomfördes en ny omgång tester med [antal] deltagare.

\textbf{Jämförelse med iteration 2:}

\begin{itemize}
    \item Medeltid för uppgift 1 minskade från XX s till YY s
    \item Framgångsgrad för uppgift 2 ökade från XX\% till YY\%
    \item Antal identifierade problem minskade från X till Y
\end{itemize}

Figur \ref{fig:comparison} visar jämförelse mellan iterationerna.

\begin{figure}[h]
    \centering
    % \includegraphics[width=0.8\textwidth]{bilder/iteration_comparison.png}
    \caption{Jämförelse av användbarhetsmetrik mellan iteration 2 och 3}
    \label{fig:comparison}
\end{figure}
