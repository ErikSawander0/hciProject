\section{Bilagor}

\textit{[Bilagor placeras efter källförteckningen och innehåller material som är relevant men för omfattande att inkludera i huvudtexten.]}


\subsection{Bilaga A: Intervjuguide}

\textit{[Exempel på innehåll i en bilaga]}

\subsubsection{Introduktion}
\begin{itemize}
    \item Hej och tack för att du deltar
    \item Presentation av projektet: [kort beskrivning]
    \item Information om att intervjun spelas in (med samtycke)
    \item Påminnelse om att deltagandet är frivilligt
    \item Uppskattat tid: [XX] minuter
\end{itemize}

\subsubsection{Bakgrundsfrågor}
\begin{enumerate}
    \item Berätta lite om dig själv och din bakgrund
    \item Hur ofta använder du [relevant system/teknologi]?
    \item ...
\end{enumerate}

\subsubsection{Huvudfrågor}
\begin{enumerate}
    \item [Fråga 1]
    \begin{itemize}
        \item Eventuella följdfrågor
    \end{itemize}
    \item [Fråga 2]
    \item ...
\end{enumerate}

\subsubsection{Avslutning}
\begin{itemize}
    \item Har du några frågor?
    \item Tack för din medverkan!
\end{itemize}


\subsection{Bilaga B: Personas}

\subsubsection{Persona 1: Karl Petterson}

\textbf{Ålder:} 28 år

\textbf{Yrke:} Mjukvara utvecklare 

\textbf{Bakgrund:} Karl bor i en lägenhet i Stockholm och jobbar hemifrån 3 dagar om veckan. Han dricker 3-4 koppar kaffe om dagen, och värdesätter konsekvens i smak och kvalitet. Han äger en espresso+maskin han sällan använder, och har intresse för innovativa tech-lösningar samt den moderna minimalistisk design populariserad av Apple. 

\textbf{Teknisk kompetens:} hög

\textbf{Mål och behov:}
\begin{itemize}
    \item En minimalistisk design. 
    \item Att det ska gå snabbt att välja kaffedryck.
    \item Att kunna smidigt anpassa kaffet. 
\end{itemize}

\textbf{Frustrationer:}
\begin{itemize}
    \item Tycker om cappuccinos och lattes, men hinner inte göra en före jobbet.
    \item Vill inte köpa en automatisk maskin, eftersom de inte passar med hans kök, dom är för silvriga och har för många fula knappar. 
    \item Kan mycket om kaffe, och föredrar att anpassa sina recept utefter sig själv och vilka böner som används, byter inte till en automatisk maskin om han inte har den friheten. 
\end{itemize}

\textbf{Citat:} \textit{"Om min kaffemaskin känns som att den är från 2005, varför skulle jag vilja använda den varje dag?"}


\subsubsection{Persona 2: Anna Bergström}
\textbf{Ålder:} 45 år
\textbf{Yrke:} Projektledare
\textbf{Bakgrund:} Anna bor i ett hus i Sundsvall med sin familj. Hon dricker 2-3 koppar kaffe om dagen, oftast på morgonen och efter lunch. Familjen har olika preferenser. Hennes partner dricker bryggkaffe, barnen vill ha mjölkdrycker, och hon själv varierar. Hon köpte en helautomatisk kaffemaskin för att alla i familjen skulle kunna göra sina egna drycker utan krångel. Hon är bekväm med teknik men vill inte spendera tid på att lära sig komplexa system.
\textbf{Teknisk kompetens:} Medel
\textbf{Mål och behov:}
\begin{itemize}
\item En maskin som alla i familjen kan använda utan att behöva fråga henne om hjälp.
\item Tydliga, enkla val som fungerar för olika användare.
\item Snabb och pålitlig - särskilt på hektiska morgnar.
\item Att kunna spara favoritinställningar för olika familjemedlemmar.
\end{itemize}
\textbf{Frustrationer:}
\begin{itemize}
\item Familjemedlemmar frågar ständigt "hur gör man en latte?" eller "vilken knapp ska jag trycka på?"
\item Nuvarande maskiner har för många alternativ som förvirrar - särskilt för barnen.
\item Svårt att komma ihåg olika inställningar för olika personer.
\item När gäster kommer över kan de inte lista ut hur maskinen fungerar.
\end{itemize}
\textbf{Citat:} \textit{"Jag vill att alla ska kunna göra sitt eget kaffe utan att jag behöver ge en manual varje gång."}

\subsubsection{Persona 3: Erik Lindqvist}
\textbf{Ålder:} 67 år
\textbf{Yrke:} Pensionerad lärare
\textbf{Bakgrund:} Erik bor i en lägenhet i Göteborg och dricker kaffe flera gånger om dagen, det är en viktig del av hans rutin. Han har artrit i händerna som gör fina motoriska rörelser utmanande, särskilt på morgonen då stelheten är värst. Han vill fortsätta vara självständig och att inte behöva be om hjälp med enkla saker som att göra kaffe. Han är bekväm med grundläggande teknik men föredrar tydliga, enkla gränssnitt.
\textbf{Teknisk kompetens:} Låg-medel
\textbf{Mål och behov:}
\begin{itemize}
\item Stora, tydliga knappar som är lätta att trycka på.
\item Gränssnitt som inte kräver precision eller fina motoriska rörelser.
\item Tydlig text och symboler som är lätta att läsa och förstå.
\item Behålla självständighet i sin vardag.
\end{itemize}
\textbf{Frustrationer:}
\begin{itemize}
\item Små touchscreen-knappar är svåra att träffa med stela fingrar.
\item Komplexa menyer med många steg är frustrerande när händerna inte samarbetar.
\item Fysiska knappar på nuvarande maskiner är för små och kräver för mycket kraft att trycka in.
\item Känner sig beroende av andra när vardagliga uppgifter blir för svåra.
\end{itemize}
\textbf{Citat:} \textit{"Jag vill kunna göra gott kaffe utan att behöva kämpa med små knappar varje morgon."}


\subsection{Bilaga C: Kompletta prototypvyer}

\textit{[Inkludera bilder på alla vyer i prototypen]}

\begin{figure}[ht]
    \centering
    % \includegraphics[width=0.8\textwidth]{bilder/prototype_view1.png}
    \caption{Prototyp - Vy 1: [Beskrivning]}
\end{figure}

\begin{figure}[ht]
    \centering
    % \includegraphics[width=0.8\textwidth]{bilder/prototype_view2.png}
    \caption{Prototyp - Vy 2: [Beskrivning]}
\end{figure}

\textit{[Fortsätt med alla vyer]}


\subsection{Bilaga D: Testuppgifter för användbarhetstester}

\textbf{Uppgift 1:}
\begin{quote}
[Beskriv uppgiften exakt som den presenterades för testdeltagarna]

Framgångskriterier: [Vad räknas som att uppgiften är slutförd?]
\end{quote}

\textbf{Uppgift 2:}
\begin{quote}
[Beskrivning]
\end{quote}

\textit{[Fortsätt med alla uppgifter]}


\subsection{Bilaga E: Samtyckesblankett}

\textit{[Exempel på samtyckesblankett om ni använt en]}

\subsubsection{Information till deltagare}

Du tillfrågas om att delta i ett projekt inom kursen Människa-datorinteraktion vid Mittuniversitetet.

\textbf{Syfte:} [Beskriv projektets syfte]

\textbf{Genomförande:} [Beskriv vad deltagandet innebär]

\textbf{Frivillighet:} Ditt deltagande är helt frivilligt och du kan när som helst avbryta utan att ange skäl.

\textbf{Konfidentialitet:} All information behandlas konfidentiellt.

\subsubsection{Samtycke}

Jag har tagit del av informationen ovan och samtycker till att delta i studien.

Datum: \rule{3cm}{0.15mm}

Underskrift: \rule{5cm}{0.15mm}

Namnförtydligande: \rule{5cm}{0.15mm}


\subsection{Bilaga F: Rådata}

\textit{[Om ni vill inkludera sammanställd rådata - var försiktig med konfidentialitet]}

Exempel:
\begin{table}[h]
\centering
\small
\begin{tabular}{|c|c|c|c|c|}
\hline
\textbf{Deltagare} & \textbf{Uppgift 1 (s)} & \textbf{Uppgift 2 (s)} & \textbf{Uppgift 3 (s)} & \textbf{SUS-poäng} \\
\hline
P1 & XX & XX & XX & XX \\
P2 & XX & XX & XX & XX \\
... & ... & ... & ... & ... \\
\hline
\end{tabular}
\caption{Rådata från användbarhetstester}
\end{table}
