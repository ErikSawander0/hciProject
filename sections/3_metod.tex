\section{Metod}

\subsection{Övergripande tillvägagångssätt}

\textit{[Beskriv er designprocess på en övergripande nivå]}

Projektet följde en iterativ användarcentrerad designprocess med följande huvudsteg:
\begin{enumerate}
    \item Målanalys och kravfångst
    \item Designalternativ och prototyper
    \item Utvärdering med målgruppen
\end{enumerate}

Totalt genomfördes [X] iterationer där designen förfinades baserat på feedback från användare.


\subsection{Iteration 1: Kravfångst och målanalys}

Baserat på tidigare erfarenheter av målgruppen och kaffemaskiner så bestämdes mål och krav. Samt analyserades befintliga lösningar för bristar.  

\subsubsection{Personas}

Baserat på tidigare erfarenheter skapades 3 personas som representerar huvudsakliga användargrupper (se Bilaga B). %MARK we need our interviews to somewhat match our personas, the third one is whatever, thats just made up shit. 


\subsection{Iteration 2: Prototyputveckling}

\subsubsection{Lo-fi prototyper}


Initialt skapades pappersprototyper och digitala skisser (lo-fi) för att snabbt kunna utforska olika designalternativ. Detta tillvägagångssätt valdes eftersom det möjliggör snabb iteration och enkelt kan kastas vid behov \cite{sharp2019}.

\textbf{Design Studio:} Gruppen genomförde en design studio-session där varje gruppmedlem skissade ett lösningsförslag var. Förslagen diskuterades och de mest lovande idéerna kombinerades. 



\subsubsection{Hi-fi prototyper}

Efter validering av lo-fi prototyper utvecklades en högfidelitetsprototyp med hjälp av html css och javascript. Prototypen inkluderade sammansatta egenskaper från lo-fi prototyperna utefter design studio resultaten.


\subsection{Iterationer : Utvärdering} % do this as we go 

\subsubsection{Användbarhetstester}

\textit{[Detta är obligatoriskt att inkludera i minst en iteration]}

För att utvärdera prototypens användbarhet genomfördes användbarhetstester med [antal] deltagare från målgruppen.

\textbf{Testupplägg:} Testerna baserades på task-based usability testing. Deltagarna fick genomföra följande uppgifter:
\begin{enumerate}
    \item Brygg en kopp kaffe
    \item Skapa en profil 
    \item Schemalägg en bryggning
    \item Skapa en profil med malningsstorlek "fine"
\end{enumerate}

\textbf{Datainsamling:} Under testerna observerades och dokumenterades:
\begin{itemize}
    \item Tid för att genomföra uppgifter
    \item Antal fel och var de uppstod
    \item Deltagarnas kommentarer (think-aloud)
    \item Ålder 
    \item Subjektiv svårighet av uppgifterna 
    \item Framgång av att utföra navigation på första försöket 
    \item Om det saknades funktionalitet
\end{itemize}

\textbf{Analysmetod:} Data analyserades genom de kvantitativa metoderna felfrekvensanalys, och deskriptiv analys, samt de kvalitativa metoderna think-aloud protokollanalys, och kategorisering av användbarhetsproblem .

\subsection{Motivering av metodval}

De valda metoderna motiveras av att de tillsammans ger en helhetsbild av användarnas behov och hur väl designen möter dessa. Intervjuerna gav kvalitativ insikt i användarnas kontext, medan användbarhetstesterna mätte konkret användbarhet.
