\section{Inledning}

I takt med att kaffekulturen växer börjar allt fler intressera sig för att brygga kaffe av kvalitet hemifrån. Dagens kaffemaskiner är ofta utrustade med avancerade funktioner som låter användaren i detalj skräddarsy sin kaffebryggning. Detta medför komplexa utmaningar inom användbarhet och tillgänglighet. 

\subsection{Bakgrund}

Efterfrågan om att kunna kontrollera små variabler som vattenpulsintervaller, malningsgrad, vattentemperatur och bryggningstid har ökat hos premium-kaffemaskiner. Denna utveckling skapar en svårighet i att skapa en bra användarupplevelse för alla användare. Ena sidan är användare som enkelt och snabbt vill kunna brygga en kopp kaffe och andra sidan entusiaster som vill ha full kontroll över alla bryggvariabler.

Många av dagens kaffemaskiner tenderar att inrikta sitt användargränssnitt till det extrema. Antingen genom att gömma avancerade funktioner eller att exponera alla detaljer. Att hitta balansen mellan enkelhet och djup funktionalitet är svårt och något som få maskiner lyckas med. Vanliga användbarhetsproblem som förekommer:

\begin{itemize}
    \item Menystrukturer med komplexa undermenyer
    \item Saknad tydlighet om vad parametrar gör
    \item Svårigheter att återanvända exakta inställningar
    \item Begränsade möjligheter till personalisering
\end{itemize}

Genom att använda principer från interaktionsdesign och användarcentrerad design kan man skapa ett gränssnitt som gör detaljerad kaffebryggning tillgängligt för en större målgrupp utan att förlora efterfrågade funktioner. 

\subsection{Syfte}

Syftet med projektet är att förbättra användarupplevelsen av avancerade kaffemaskiner genom skapandet av ett gränssnitt som tilltalar både nybörjare och erfarna kaffeentusiaster samtidigt som den sänker inlärningströskeln.  

Projektet syftar till att utforska hur ett profilsystem kan användas för att undanhålla komplexitet på ett sätt så användare själv kan anpassa komplexitetsnivån utifrån intresse.

\subsection{Mål}

Projektets mål är att leverera:

\begin{itemize}
    \item En högfidelitetsprototyp av ett touchbaserat gränssnitt för en premium-kaffemaskin som balanserar enkelhet med avancerade funktionaliteter.
    \item Ett profilbaserat system där användare kan:
    \begin{itemize}
        \item Brygga kaffe med förinställda profiler
        \item Skapa bryggprofiler med egna anpassade preferenser. 
        \item Anpassa avancerade inställningar (vattenpulsinervall, temperatur, malningsgrad och extraktionstid)
    \end{itemize}

    \item Personaliseringsfunktion i form av val av färgtema

    \item Dokumentation och utvärdering av designprocessen genom användarcentrerade metoder
    
\end{itemize}


\subsection{Avgränsningar}

Projektet avgränsas till följande:

\begin{itemize}
    
    \item \textbf{Plattform och implementation:} Projektet fokuserar på design av användargränssnittet (UX/UI). Funktioner kommer interaktivt vara funktionella med inte kopplade till praktisk funktionalitet. Inga back-end system kommer att implementeras.

    \item \textbf{Maskintyp:} Fjärrstyrning eller mobilapplikationer ingår inte i projektet. Fokus ligger på gränssnittet på kaffemaskinens touchskärm.

    \item \textbf{Språk:} För att förenkla utvecklingsprocessen presenteras avandargränssnittet endast på engelska.

    \item \textbf{Målgrupp:} Kaffeälskare med varierande erfarenhet av kaffebryggning som vill göra premium kaffe hemifrån.


\end{itemize}

\textbf{Motivering:} Projektets avgränsningar möjliggör att fokus kan ligga på användbarhet och interaktionsdesign utan att arbetet belastas av tekniska implementationer. Detta resulterar i en mer iterativ och användarcentrerad process inom given tidsram. 

\textbf{Potentiella konsekvenser:} Eftersom ingen backend-implementation utförs begränsas möjligheten till feedback kring responstider. Detta påverkar hur realistiskt användbarhetstesterna visar den faktiska produkten. 


\subsection{Arbetsfördelning}

Projektet genomförs som ett samarbete där alla gruppmedlemmar bidrar till designprocessen. Följande arbetsfördelning har etablerats:

\begin{table}[h]
\centering
\begin{tabular}{|l|p{9cm}|}
\hline
\textbf{Gruppmedlem} & \textbf{Ansvar och bidrag} \\
\hline
Elias Danielsson & Brainstorming design-idéer, prototyputveckling (initial fas), rapportskrivande (generell struktur, inledning), feedback insamling, merging av prototyper (Högnivå), förbättring av prototyp \\
\hline
Erik Sawander & Brainstorming design-idéer, prototyputveckling (initial fas), rapportskrivande(fill in later), feedback insamling, \\
\hline
Theodor Christensen & Prototyputveckling (initial fas), Skrivande av frågeformulär, rapportskrivande(fill in later), feedback insamling, \\
\hline
Victor Hillström & Prototyputveckling (initial fas), merging av prototyper (Lågnivå), rapportskrivande(fill in later), feedback insamling, \\
\hline
\end{tabular}
\caption{Arbetsfördelning i projektgruppen (uppdateras löpande under projektets gång)}
\end{table}
